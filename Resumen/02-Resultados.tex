\section{Resultados}

\subsection{Topología del espectro de autovalores}
En la Figura \ref{FigureIsosurfaces} se aprecian las isosuperficies del espectro de autovalores para
una celda comprimida hasta  68.04$\AA^3$ (izquierda) , para la celda en equilibrio (centro), y para una celda
expandida hasta 70.25$\AA^3$.  
En la celda comprimida, las bandas 23 y 24 son las que tienen sus máximos en el nivel de Fermi, 
y se eligió una isosuperficie para un valor 0.1 eV por debajo de dicho valor para apreciar sus vasijas de
máximo, que se forman en los puntos $Z (0, 0, \pm0.5)$ y $A (\pm0.5,\pm 0.5, \pm0.5)$. Por su parte, la 
banda 25 tiene su mínimo sobre el punto $\Gamma$ de la BZ, y se escogió una isosuperficie 0.2eV por 
encima de de la $E_{cbm}$. El comportamiento de la banda de conducción no se altera para 
el resto de los volumenes si bien se observará mas adelante que las masas efectivas de los electrones 
cambian considerablemente. La multiplicidad de la vasija en $Z$ es de 2 mientras que las
vasija en $A$ y $\Gamma$ son simples.

\begin{figure}
  \center %  \includegraphics[width=0.32\textwidth]{../ZrO2-04-merged-23-24-25.png}
  \includegraphics[width=0.32\textwidth]{../ZrO2-04-merged.pdf}
  \includegraphics[width=0.32\textwidth]{../ZrO2-05-merged.pdf}
  \includegraphics[width=0.32\textwidth]{../ZrO2-07-merged.pdf}
  \caption{\protect\label{FigureIsosurfaces}
  isosuperficies  para una estructura en compresión, equilibrio y expansión}
\end{figure}

Por otro lado, para el volumen de equilibrio se observa que nuenvamente las bandas  23 y 24 
tienen sus máximos sobre el nivel de fermi, pero la banda 24 a demas de los puntos $Z$ y $A$
presenta una vasija en el punto medio entre $\Gamma$ y $M (\pm0.5, \pm0.5, 0)$ con una 
elongación que puede considerarse sobre el segmento $X (\pm0.5, 0, 0) \rightarrow X' (0, \mp0.5, 0)$ con 
multiplicidad de 4. 

Por último para las celdas expandidas, se observa que la banda 24 es la única observada en 
la zona de valencia con una única vasija de máximo en el punto medio entre $\Gamma$ y $M$ de 
similares características a lo descrito anteriormente.

\subsection{Estructuras de Bandas}

Con la información de la evolución de la topología del espectro de autovalores en 
la BZ, se diseña un camino entre los puntos de simetría de la BZ para obtener las 
relaciones  $E(\mathbf{k}$ necesarias para el cálculo de las masas transversales y
longitudinales en cada vasija. Se toma un camino continio $\Gamma \rightarrow X \rightarrow
M \rightarrow \Gamma \rightarrow Z \rightarrow A \rightarrow M$, al cual se le agrega el segmento 
$ X \rightarrow X' $ para obtener la masa longitudinal de la vasija de la banda de valencia 
en las estructuras expandidas, como se ilustra en la figura \ref{FiguraKPath}.

\begin{figure}
  \centering
  \includegraphics[width=0.45\textwidth]{../kpath-a.png}
  \includegraphics[width=0.45\textwidth]{../kpath-xx.png}
  \caption{\label{FiguraKPath} Puntos K recorridos para la estructura de bandas}
\end{figure}

Puede observarse en todos los casos que el mínimo de la banda de conducción
se encuentra en $\Gamma$ para la banda 25. 
Puede observarse en la Figura \ref{FigureStructureBandWall} que en compresión
el valence band maximum se encuentra en el punto $A$ con cercanía del máximo 
local en $Z$, pero al llegar al equilibrio hay una cresta en el punto medio entre 
$M$ y $\Gamma$  que se convertirá en el máximo en expansión. Notar que la cresta entre 
$X$ y $X'$ corresponde al mismo punto, la primer cresta la componente transversal y 
la última la longitudinal de la masa efectiva en este punto.

\subsection{Masas Efectivas}

Utilizando las relaciones $E(\mathbf{k})$ obtenidas, se utiliza el código 
sumo \cite{MGanose2018} para calcular las masas efectivas sobre 
cada segmento. Se utiliza un ajuste no parabólico con cinco puntos por cada 
segmento para obtener los valore de masa que se observan en la tabala

\begin{table}
  \caption{
    \protect\label{TableEffMassVol} Tabla Masas efectivas en función del volumen para electrones 
    y huecos. Cuando se indican dos valores es porque SUMO calcula desde el máximo hacia 
    el vértice, y debe tomarse el promedio.
  }

    \small
    \centering
  \begin{tabular*}{0.8\textwidth}{c|c|c|c|c|c}
    \hline
vol&$m_{h, long}$ &$m_{h, trans}$ &multiplicidad& $m_{e,long}$ &$m_{e,trans}$ \\
       &$X - X'$&  $M - \Gamma$ &    &$\Gamma - Z$ &$\Gamma -  X$\\
     \hline
70.12  & 11.42       & 2.790  &  4  &   2.26   & 0.65 \\
72.25  & 9.15        & 2.498  &  4  &   2.35   & 0.65 \\ 
74.41  & 8.36        & 2.514  &  4  &   2.44   & 0.65 \\ 
76.62  & 7.18        & 2.210  &  4  &   2.49   & 0.64 \\
78.88  & 6.92        & 2.286  &  4  &   2.59   & 0.64 \\
81.17  & 6.58        & 2.101  &  4  &   2.70   & 0.63 \\
\hline
% ( 0,629  +0,671)/2
% ( 0,630  +0,671)/2
% ( 0,674  +0,635)/2
% ( 0,623  +0,659)/2
% ( 0,625  +0,659)/2
% ( 0,653  +0,619)/2
%  &( -2.782  -2.799)/2
%  &( -3.016  -1.980)/2
%  &( -2.386  -2.643)/2
%  &( -2.076  -2.344)/2
%  &( -3.086  -1.487)/2
%  &( -2.234  -1.968)/2
  \end{tabular*}
\end{table}

\begin{figure}

  \includegraphics[width=\textwidth]{../make_valenceband_wall.pdf}
  \caption{\label{FigureStructureBandWall} Evolucion de la estructura de bandas
  en la zona de valencia en función del volúmen El volumen 5 corresponde a 
  la estructura en equilibrio. Los volumenes entre 1 y 4 son en compresión 
  mientras que de 6 a 9 son en expansión}
\end{figure}




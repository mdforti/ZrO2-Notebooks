\section{Método}

La t-ZrO\textsubscript{2} tiene una estructura cristalina en el grupo 
P42/nmc \footnote{https://materialsproject.org/materials/mp-2574/}. Sus 
propiedades de equilibrio han sido ampliamente estudiadas en forma 
computacional y experimental \cite{Milman2009, Garcia2006}. Sin embargo
pretendemos aquí obtener una estimación de la variación de las masas 
efectivas con la temperatura.

Para ubicar las basijas de los extremos en la zona de valencia y de conducción
del espectro de autovalores en la BZ, se utilizó la Teoría de la funcional densidad (DFT)
\cite{PhysRev.136.B864, PhysRev.140.A1133} como se implementa en el código VASP \cite{doi:10.1002/jcc.21057}.
La zona de Brillouin se discretiza usando una grilla de MP incluyendo el punto $\Gamma$, y se varía la 
densidad de la misma hasta asegurar la convergencia del espectro de autovalores, observando que
la energía del máximo de la banda de valencia $E_{vbm}$ y del mínimo de la banda de conducción 
$E_{cbm}$ converjan \footnote{Ver que criterio se puede poner con lo que hay hecho},
lo cual se alcanza de MP\cite{PhysRevB.13.5188}
de 17x17x15. Se utiliza el método del tetraedro con correcciones de Blöchl\cite{Blochl1994} 
para las integraciones en la zona de Brillouin.

\subsection{Basijas de las bandas de valencia y conducción}
Los cálculos del espectro del autovalores se repiten para diez volúmenes 
entre 60.12 y 81.17 $\AA^3$ que se obtienen deformando la celda isostáticamente y conservando
las relaciones de aspecto entre los vectores de red. En todos los casos se grafican las 
isosuperficies de las bandas de valencia y de conducción en la BZ. Para la banda de conducción, se fija 
el nivel de la isosuperficie alrededor de 0.1 eV debajo del nivel de fermi. Para la banda de valencia, 
se fija la isosuperficie 0.2eV por sobre el $E_{cbm}$
Para construir las isosuperficies se utilizan los códigos \texttt{c2x}\cite{Rutter2018}
para transformar la salida de VASP de manera que puede ser leída por el código de visualización
\texttt{xcrysden}\cite{Kokalj1999}.
en unde manera de identificar las basijas de máximo,

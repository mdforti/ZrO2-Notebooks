\section{Introducción}
Las masas efectivas de los electrones en un cristal pueden calcularse
a partir de la relación de dispersión de su estructura de bandas, y depende 
en general del mathbftor de onda. En los
semiconductores más sencillos puede aproximarse esta relación 
en las proximidades de los extremos por una
relación parabólica
\begin{equation}\label{EqParabolica}
  E(\mathbf{k}) = E_0 + \dfrac{\hbar (\mathbf{k}-\mathbf{k_0})^2}{2 m^{*}}
\end{equation}
donde E es la energía de la banda para un punto $\mathbf{k}$ de la red recíproca, $\mathbf{k_0}$
es 
el punto donde se alcanza el extremo $E_0$ para la banda en cuestión, $\hbar$ es la constante de 
Plank y $m^{*}$ es la masa en reposo del electrón. 

Debe notarse sin embargo que dicha expresión supone isotropía respecto de la posición 
de los extremos. En materiales anisotrópicos, la masa efectiva se considera una cantidad
tensorial y suele calcularse en función del mathbftor de onda,

\begin{equation}\label{EqTensorDefinition}
  m_{ij}^{-1} = \frac{1}{\hbar}\dfrac{\partial^2 E}{\partial k_i  \partial k_j}
\end{equation}

En un problema real puede considerarse que las masas de los huecos pueden calcularse
en el máximo de la banda de valencia más alta y la masa del electrón en el mínimo de la banda
de conducción. 

Los problemas anisotrópicos más sencillos permiten aproximar al tensor de masas por sus
componentes longitudinal y transversal\footnote{ https://www.iue.tuwien.ac.at/phd/mwagner/diss.html}
en las direcciones principales de la red recíproca. 
El problema radica entonces en ubicar correctamente los valles de energía dentro de la BZ
para calcular las relaciones $E(\mathbf{k})$ necesarias para aproximar las masas longitudinal 
$m_l(\mathbf{k})$ y transversal $m_t(\mathbf{k})$.

\subsection{concentración de portadores en la circonia}

En la Circonia, el equilibrio entre las  concentraciones de portadores 
y de defectos puntuales determinará sus propiedades de transporte a escala macroscópica.  Las
mismas pueden ser de vital importancia para determinar la capacidad del óxido protector
de las aleaciones de Zr o incluso en dispositivos electrónicos basados en ZrO2. En particular
la Circonia tetragonal (t-ZrO\textsubscript{2}) juega un papel primordial en la resistencia
a la corrosión de las aleaciones de uso nuclear. 
La concentración de portadores de carga puede calcularse en función de la temperatura y la densidad
de estados efectiva de cada portador en la posición de sus extremos. 

\begin{equation}\label{EqPortadoresT}
  n_P = N_p \exp {\Bigl( -\frac{\Delta E_P}{k_B T} \Bigr)}
\end{equation}
donde $K_B$ es la constante de Boltzmann, $Delta E_P$ es la diferencia de energía 
entre el extremo de la banda y la energía de Fermi, y  $N_P$ es la densidad de
estados efectiva del portador, 

\begin{equation}\label{EqCarrEffDens}
  N_P = \Bigl( \frac{2 \pi m_{P}^{*} k_B T }{h} \Bigr)^{\frac{3}{2}}
\end{equation}
en donde se utiliza la masa efectiva del portador. Ahora bien, si las masas efectias pueden 
aproximarse por sus componentes transversal y longitudinal, la masa efectiva 
puede aproximarse por la masas de la densidad de estados y teineindo en cuenta la 
multiplicidad $g$ de la basija correspondiente al extremo de la banda, 

\begin{equation}\label{EqDOSEffMass}
  \sqrt[3]{g_P^2 m_t ^2 m_l}
\end{equation}

Por último, entonces, la concentración intrínseca de cada portador $P$ puede tenerse como

\begin{equation}\label{EqFullCarrierConcentration}
  n_P = N_P \exp{ \Bigl( -\frac{\Delta E_P}{k_B T} \Bigr) } 
\end{equation}

